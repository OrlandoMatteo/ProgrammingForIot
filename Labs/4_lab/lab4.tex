% Created 2020-11-12 gio 11:01
% Intended LaTeX compiler: pdflatex
\documentclass{article}
\usepackage[utf8]{inputenc}
\usepackage[T1]{fontenc}
\usepackage{graphicx}
\usepackage{grffile}
\usepackage{longtable}
\usepackage{wrapfig}
\usepackage{rotating}
\usepackage[normalem]{ulem}
\usepackage{amsmath}
\usepackage{textcomp}
\usepackage{amssymb}
\usepackage{capt-of}
\usepackage{hyperref}
\usepackage[sfdefault]{cabin}
\usepackage[T1]{fontenc}
\author{Matteo Orlando}
\date{}
\title{Programming for Iot - Lab 4}
\hypersetup{
 pdfauthor={Matteo Orlando},
 pdftitle={Programming for Iot - Lab 4},
 pdfkeywords={},
 pdfsubject={},
 pdfcreator={Emacs 26.3 (Org mode 9.4)}, 
 pdflang={English}}
\begin{document}

\maketitle
\tableofcontents


\section{Intrduction}
\label{intrduction}
In this lab we will try to develop different kind of sensor simulator
that use the MQTT protocol

Remember to always use SenML as data format:

\begin{verbatim}
    {
        "bn": "http://example.org/sensor1/", 
        "e": [
                {
                    "n": "temperature", 
                    "u": "Cel", 
                    "t": 1234, 
                    "v":22.5 
                } 
             ]
    }
\end{verbatim}

\subsection{Exercise 1}
\label{exercise-1}
Develop an MQTT publisher to emulate a temperature sensor that publish
random values in the range [--10,39] every 5 seconds for 2 minutes.
Develop also an MQTT subscriber that receives these values, prints these
on screen and save these on a json file called \emph{temp\textsubscript{log.json}}. To
generate the values, you can use one of the functions of the library
random listed below:

\begin{itemize}
\item random.randint(a,b)

\item random.random()

\item random.uniform(a,b)
\end{itemize}

\subsection{Exercise 2}
\label{exercise-2}
Develop an MQTT publisher to emulate an heartrate sensor that publish
random values in the range [55,180] every 5 seconds for 2 minutes.
Develop also an MQTT subscriber that receives these values, prints these
on screen and save these on a json file called \emph{hr\textsubscript{log.json}}. To
generate the values, you can use one of the functions of the library
numpy listed at
\href{https://docs.scipy.org/doc/numpy-1.15.0/reference/routines.random.html}{this
link}, try with different functions to evaluate which is the one that
has the most realistic result. You can find a visualization of some of
this generator
\href{https://colab.research.google.com/drive/1JsxjaRDYnoMb6dQ5MZsLKH7ZDy7QzH9O?usp=sharing}{here}
(you can even make your test there before writing your code)

\subsection{Exercise 3}
\label{exercise-3}
Using the file \emph{hr\textsubscript{log.json}} created in the exercise before, develop an
MQTT publisher to emulate a heartrate sensor that publish on by one the
data in that file

\subsection{Exercise 4}
\label{exercise-4}
Develop an MQTT publisher to emulate a sensor of your choice that
publish random values in the 3 possible ranges. For example for an
heartrate sensor we could define 3 ranges as [resting,sport,danger].
Then create a simple terminal client for this MQTT publisher to select
the range to be used to send the data. Develop also an MQTT subscriber
to receive those data and in case of data in a warning range provide a
visual feedback.
\end{document}
